\documentclass[10pt,twocolumn,letterpaper]{article}

\usepackage{cvpr}
\usepackage{times}
\usepackage{epsfig}
\usepackage{graphicx}
\usepackage{amsmath}
\usepackage{amssymb}
\usepackage{bbm}
% Include other packages here, before hyperref.

% If you comment hyperref and then uncomment it, you should delete
% egpaper.aux before re-running latex.  (Or just hit 'q' on the first latex
% run, let it finish, and you should be clear).
\usepackage[breaklinks=true,bookmarks=false]{hyperref}

\cvprfinalcopy % *** Uncomment this line for the final submission

\def\cvprPaperID{****} % *** Enter the CVPR Paper ID here
\def\httilde{\mbox{\tt\raisebox{-.5ex}{\symbol{126}}}}

% Pages are numbered in submission mode, and unnumbered in camera-ready
%\ifcvprfinal\pagestyle{empty}\fi
\setcounter{page}{1}
\begin{document}

%%%%%%%%% TITLE
\title{A survey of Knowledge Distillation}

\author{Trung Dao\\
VinAI Research\\
14 Thuy Khue, Tay Ho, Ha Noi\\
{\tt\small v.trungdt21@vinai.io}
}

\maketitle
%\thispagestyle{empty}

%%%%%%%%% ABSTRACT
\begin{abstract}
   For the last few years, deep learning has proved its ability to solve complicated problems that were too cumbersome to handle. It is undeniable that key factors contributing to this sucess are mainly large-scale data and deep neural networks with billions of parameters. However, to deploy and smoothly run these enormous models on on-edge devices is not always a straightforward process. In order to overcome this, a variety of model compression and acceleration techniques have been developed. As one of the outstand techniques, knowledge distillation (KD) seeks to transfer information learned from one large model (called \textit{teacher}) to smaller one (called \textit{student}). This survey firstly gives a brief overview of what KD is then quickly investigates some of the recent work about training schemes and teacher-student architectures.
\end{abstract}

%%%%%%%%% BODY TEXT
\section{Introduction}

After achieving extraordinary establishment in various fields of Machine Learning using neural networks, scientists began to move interests into network compresssion and enhancement. Several worth-mentioned approaches which seek to create smaller model and cost-efficient are: parameter pruning/sharing, model quantization, low-rank factorization and knowledge distillation. A comprehensive overview on these approaches is outside of the scope of this survey, and the focus of this paper is solely about knowledge distillation, which has attracted many attentions from the deep learning research community in the recent years. 

Comparing to other compression methods, KD might be considered superior since it can squeeze down a network disregarding the structural difference between the teacher and student network. The original idea of KD was proposed in \cite{firstkdpaper}, where the author focused on transfering the information from a large model or an ensemble of models into a smaller model without performance loss. This work is later generalized and popularized by Hinton et al. \cite{hintonfirstkd}. In their work, they found out an astounding observation that it is simplier to train a smaller-scale network (classifier) using the soften predictions of another classifier as target value rather than the ground truth (one-hot) labels and later on called this procedure as \textit{distillation} (short for knowledge distillation). They also stated that these probabilities offer richer information than labels alone, and empirically help the student network learn better. Knowledge distillation field since then has been developed and applied in various forms, but the main characteristic of KD is still remained the same: Teacher-Student framework, where the teacher model provides useful \textit{knowledge} to the student model in order to improve its learning performance.

Thanks to its effective on neural network compression and acceleration, KD has been widely applied in different fields of artificial intelligence: visual/speech recognition, natural language processing (NLP). In visual recognition, KD was firstly applied on classification tasks \cite{hintonfirstkd,visualtask01,visualtask02,visualtask03,visualtask04} then later on expanded to other applications such as image segmentation \cite{segment01}, lane detaction \cite{lanedetect01}, facial landmark detection \cite{facial01, facial02}. In other field such as NLP, KD also proves its worth as being used to compress complex structures such as BERT \cite{nlp01, nlp02}. To briefly generalized, KD offers not only lightweight deep models, which allows deploying models to on-edge devices efficiently but recently also competive performant ones.

The main contributions of this paper consist of:
\begin{enumerate}
   \item Provide a comprehensive overview of Knowledge Distillation: problem definition, type of knowledge, and its recent progress.
   \item Investigate how research community try to theoretically explain KD.
   \item Offer several scenarios and its feasible solution based on other recent works.
\end{enumerate}

\section{Background: Knowledge Distillation}

\subsection{Problem statement}
Neural network models have been successful in a myriad of fields including extremely complex problems. However, these models are enormous in size and computational hungry, thus cannot be deployed to on-edge devices or simply not feasible in some situations. To compensate this situation, knowledge distillation was first proposed in \cite{firstkdpaper} and properly formalize in \cite{hintonfirstkd}. Buciluǎ et al. \cite{firstkdpaper}, the knowlede is transferred from the larger model to smaller model while minimizing the logits difference produced by those two models respectively. This vanilla method open the first path of knowledge type that can be use to distilled, later on, there started to have some other works using the activations, feature maps of intermediate layers as the guide for the student network \cite{featurebased01, featurebased02_AT,featurebased03_relu}. Some other methods to extract knowledge by comparing relationship between difference layers of teacher model \cite{relbase01, relbase02} are also worth mentioning but won't be discussed in this survey.

%-------------------------------------------------------------------------
\subsection{Knowledge}
\subsubsection{Knowledge from logits}
In this type of knowledge extraction, it usually refers to using the neural response of the final output layer, or also known as \textit{logits} of the teacher model. As stated before, the first application of logits in KD is used in \cite{firstkdpaper}, but in many situations, given a highly confident teacher model, the output of softmax function gives more less the same information as the ground truth label (since this is the core idea of softmax function - maximizing the probabilities of one while minimizing the others). Tackling this problem, Hinton et al. \cite{hintonfirstkd} proposed the concept of 'soft labels' and declared that this type of label contains informative \textit{dark knowledge}. Given the \textbf{logits} $z$ from a network, the 'soft label' $p_i$ of an image is defined as:
\[
   p_i = \frac{\text{exp}(\frac{z_i}{\rho})}{\sum_j \text{exp}(\frac{z_i}{\rho})}
\]
where $\rho$ is the temperature parameter, notice that when $\rho=1$, we get the normal softmax function. Determine which value is optimal for $\rho$ is still a debatable topics but it is argued that while increasing $\rho$, the label becomes softer and providing more information about which class is similar to the predicted label. Accordingly, the logits-based distillation loss function is defined as follow:
\[
   \mathcal{L}_{\text{LB}}(x;\theta) = \alpha * \mathcal{H}(y, \sigma(z_s)) + \beta * \mathcal{H}(\sigma(z_t;\rho), \sigma(z_s;\rho))
\]
where $x$ is the input, $\theta$ are the student model weights, $\mathcal{H}(.)$ is the cross-entropy loss function, $y$ is the ground truth label, $\sigma(.)$ is the $\rho$-parameterized softmax function, $\alpha, \beta$ are the coefficients to tune in order to balance between both cross-entropies, and $z_t, z_s$ are the logits of teacher and student model respectively.

Logits-based knowledge is fairly straightforward and easy to implement so it is often the first approach used when need to apply knowledge distillation. As far as the survey goes, there are two main motives of using logits-based knowledge: 1) using soft labels 2) create/use noisy data to train. A brief table containing its description and several related work can be found at \ref{tab:logitbased_related}. Nevertheless, since this type of knowledge only utilize the final output of the teacher model, it fails to offer information gathered in the intermediate layers, which later was proved to contain many informative result \cite{featurebased01}. Other than that, due to its characteristic, logits-based knowledge is bounded with the supervised learning.

\subsubsection{Knowledge from intermediate layers}
What's extraordinary about deep neural networks is how they extract and abstractly represent features through the feature maps generated between each layer. Hence, feature-based knowledge would theoretically contain richer information than logits-based knowledge. Not only so, by combining both intermediate and last layer's output, we can consider feature-based knowledge is an extension of the previous knowledge type.

Romero et al. \cite{featurebased01} introduced the term \textit{hint} as the outputs of a teacher hidden layer, which are used to guide the student learning process. The core idea is to choose some intermediate layers in both teacher and student layer and force the student to mimics the result of the teacher's feature maps. Motivated by this work, there has been many studies to investigate which hint layer/ guided layer to choose and how to measure the distance between them. The feature-based distillation loss function is often defined as follow:
\[
      \mathcal{L}_{FB}(f_t, f_s) = \mathcal{D}(\Phi_t(f_t), \Phi_s(f_s))
\]
where $f_t, f_s$ are the selected hint and guided layers, $\Phi_t, \Phi_s$ are the transformation functions for the mentioned layers respectively, which often are applied when the result of two layers are mismatched, $\mathcal{D}$ is the distance function measuring the similarity between the hint and guided layer. L1, L2 is often used for this distance function, but for some works, more complex function might be used such as the Maximum Mean Discrepancy (MMD) metric \cite{featurebased04_mmd} or Kullback–Leibler divergence \cite{featurebased05_kl}.

Eventhough this knowledge type offers more favorable information for the student learning process, where to pick the hint/ guided layers or the transformation functions still need more investigations. For examples, the teacher's layer transformation function are directly critical for this process since there are risks of losing information in the process of transforming. Several past work has already encountered this problem since the feature map's dimension got downscaled, causing loss of knowledge \cite{featurebased02_AT,featurebased06_meal}. To counter this situation, one could either come up with novel transformation function such as margin RELU\cite{featurebased03_relu} or not use any transformation at all \cite{featurebased01}. But as investigated by Heo et al. \cite{featurebased03_relu}, the hints consist of both beneficial and adverse information so we should try to only gather the positive ones rather than distilling all of it. The same goes for the guided layers, sometimes researchers use the same transformation \cite{featurebased02_AT}, which might lead to the same information loss. Some other works use $1 \times 1$ convolutional layer as a student transform \cite{featurebased01,featurebased03_relu} in order to match the depth dimension with the hints, so no information would be lossed during the process. A more detail of progressive work using feature-based knowledge can be found at figure \ref{tab:featbased_related}.
\section{Explaining Knowledge Distillation}
As discussed in the previous sections, knowledge distillation has been successfully applied in various fields and applications. However, clarying how and why KD works in general and its models are usually superior than those trained from raw data still remains an work in progress.
\section{Regularizing Knowledge distillation}
As one of the most beneficial characteristic of KD, any student can learn from any teacher disregarding the structural difference. But it is emperically proved that well-trained large DNN doesn't often make good teachers due to the mismatched capacity, which makes the student unable to mimic it \cite{complexitygap}. 

To tackle this problem, multiple work shares the same idea of regularizing the teacher model \cite{complexitygap, labelsmoothingnoise}. In \cite{complexitygap}, Cho et al. proposed a new process ESKD (Early-stopped knowledge distillation) after emprically prove sequential knowledge distillation is also not that efficient since it can only outperform one model training from scratch but not ensemble of those. They argued that the found solution space of the teacher is not accessible from the student, which means to find a teacher whose discovered solution should be discoverable by the student. Based on another works, the author assumed early stopping allows large model to behave as a small network while still having better search space than smaller ones. Different from Cho et al., Lukasik et al. \cite{labelsmoothingnoise} investigates the denoising effect of label smoothing on noisy data then applies it on the distillation process in order to test its effectiveness. The ending result is that applying label smoothing on the teacher significantly enhances over vanilla distilaltion, while applying the same on the student has mixed results.

Having the same idea, Li et al. \cite{teacherfree} investigates and compares KD with label smoothing regularizer and later on proposed a novel Teacher-free Knowledge distillation (TfKD) framework. It started with the observation that using either poorly trained teacher to distil student or student to teach teacher model still can improve and enhanced the guided models, which suggests the author to consider KD as a regularization term (strongly related to label smoothing regularizer). Having that in mind, the paper consider replacing the class distribution predictions of teacher model with a simppler one, implemented in the TfKD framework. This framework is particularly useful in circumstances where a more efficient teacher model is inaccessible or where only minimal computing resources are available. There are two methods of distilling in TfKD framework: 1) self-training distillation and 2) Combine KD with Label Smoothing Regularizer to create a 100\% accuracy teacher. Both of the methods are very simple yet effective and also emperically proved to be performant.
\section{Conclusion}
The main technical information and applications of knowledge distillation have been covered in this survey.
We also briefly review the taxonomy methods for current KD approaches and include description of the problem. We then investigate how KD is perceived and explained in the current past work. Finally some methods of regularizing KD while applying in real-worl application are also discussed.

{\small
\bibliographystyle{ieee_fullname}
\bibliography{egbib}
}
\end{document}
