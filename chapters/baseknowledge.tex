\section{Background: Knowledge Distillation}

\subsection{Problem statement}
Neural network models have been successful in a myriad of fields including extremely complex problems. However, these models are enormous in size and computational hungry, thus cannot be deployed to on-edge devices or simply not feasible in some situations. To compensate this situation, knowledge distillation was first proposed in \cite{firstkdpaper} and properly formalize in \cite{hintonfirstkd}. Buciluǎ et al. \cite{firstkdpaper}, the knowlede is transferred from the larger model to smaller model while minimizing the logits difference produced by those two models respectively. This vanilla method open the first path of knowledge type that can be use to distilled, later on, there started to have some other works using the activations, feature maps of intermediate layers as the guide for the student network \cite{featurebased01, featurebased02_AT,featurebased03_relu}. Some other methods to extract knowledge by comparing relationship between difference layers of teacher model \cite{relbase01, relbase02} are also worth mentioning but won't be discussed in this survey.

%-------------------------------------------------------------------------
\subsection{Knowledge}
\subsubsection{Knowledge from logits}
In this type of knowledge extraction, it usually refers to using the neural response of the final output layer, or also known as \textit{logits} of the teacher model. As stated before, the first application of logits in KD is used in \cite{firstkdpaper}, but in many situations, given a highly confident teacher model, the output of softmax function gives more less the same information as the ground truth label (since this is the core idea of softmax function - maximizing the probabilities of one while minimizing the others). Tackling this problem, Hinton et al. \cite{hintonfirstkd} proposed the concept of 'soft labels' and declared that this type of label contains informative \textit{dark knowledge}. Given the \textbf{logits} $z$ from a network, the 'soft label' $p_i$ of an image is defined as:
\[
   p_i = \frac{\text{exp}(\frac{z_i}{\rho})}{\sum_j \text{exp}(\frac{z_i}{\rho})}
\]
where $\rho$ is the temperature parameter, notice that when $\rho=1$, we get the normal softmax function. Determine which value is optimal for $\rho$ is still a debatable topics but it is argued that while increasing $\rho$, the label becomes softer and providing more information about which class is similar to the predicted label. Accordingly, the logits-based distillation loss function is defined as follow:
\[
   \mathcal{L}_{\text{LB}}(x;\theta) = \alpha * \mathcal{H}(y, \sigma(z_s)) + \beta * \mathcal{H}(\sigma(z_t;\rho), \sigma(z_s;\rho))
\]
where $x$ is the input, $\theta$ are the student model weights, $\mathcal{H}(.)$ is the cross-entropy loss function, $y$ is the ground truth label, $\sigma(.)$ is the $\rho$-parameterized softmax function, $\alpha, \beta$ are the coefficients to tune in order to balance between both cross-entropies, and $z_t, z_s$ are the logits of teacher and student model respectively.

Logits-based knowledge is fairly straightforward and easy to implement so it is often the first approach used when need to apply knowledge distillation. As far as the survey goes, there are two main motives of using logits-based knowledge: 1) using soft labels 2) create/use noisy data to train. A brief table containing its description and several related work can be found at \ref{tab:logitbased_related}. Nevertheless, since this type of knowledge only utilize the final output of the teacher model, it fails to offer information gathered in the intermediate layers, which later was proved to contain many informative result \cite{featurebased01}. Other than that, due to its characteristic, logits-based knowledge is bounded with the supervised learning.

\subsubsection{Knowledge from intermediate layers}
What's extraordinary about deep neural networks is how they extract and abstractly represent features through the feature maps generated between each layer. Hence, feature-based knowledge would theoretically contain richer information than logits-based knowledge. Not only so, by combining both intermediate and last layer's output, we can consider feature-based knowledge is an extension of the previous knowledge type.

Romero et al. \cite{featurebased01} introduced the term \textit{hint} as the outputs of a teacher hidden layer, which are used to guide the student learning process. The core idea is to choose some intermediate layers in both teacher and student layer and force the student to mimics the result of the teacher's feature maps. Motivated by this work, there has been many studies to investigate which hint layer/ guided layer to choose and how to measure the distance between them. The feature-based distillation loss function is often defined as follow:
\[
      \mathcal{L}_{FB}(f_t, f_s) = \mathcal{D}(\Phi_t(f_t), \Phi_s(f_s))
\]
where $f_t, f_s$ are the selected hint and guided layers, $\Phi_t, \Phi_s$ are the transformation functions for the mentioned layers respectively, which often are applied when the result of two layers are mismatched, $\mathcal{D}$ is the distance function measuring the similarity between the hint and guided layer. L1, L2 is often used for this distance function, but for some works, more complex function might be used such as the Maximum Mean Discrepancy (MMD) metric \cite{featurebased04_mmd} or Kullback–Leibler divergence \cite{featurebased05_kl}.

Eventhough this knowledge type offers more favorable information for the student learning process, where to pick the hint/ guided layers or the transformation functions still need more investigations. For examples, the teacher's layer transformation function are directly critical for this process since there are risks of losing information in the process of transforming. Several past work has already encountered this problem since the feature map's dimension got downscaled, causing loss of knowledge \cite{featurebased02_AT,featurebased06_meal}. To counter this situation, one could either come up with novel transformation function such as margin RELU\cite{featurebased03_relu} or not use any transformation at all \cite{featurebased01}. But as investigated by Heo et al. \cite{featurebased03_relu}, the hints consist of both beneficial and adverse information so we should try to only gather the positive ones rather than distilling all of it. The same goes for the guided layers, sometimes researchers use the same transformation \cite{featurebased02_AT}, which might lead to the same information loss. Some other works use $1 \times 1$ convolutional layer as a student transform \cite{featurebased01,featurebased03_relu} in order to match the depth dimension with the hints, so no information would be lossed during the process. A more detail of progressive work using feature-based knowledge can be found at figure \ref{tab:featbased_related}.